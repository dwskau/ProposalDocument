\bodychapter{FUTURE WORK}
\label{futureWork}
This pass at implementing the visualization primitives model has proven to be effective at some things, however there are still limitations due to the implementation.
There are several areas that need to be re-examined in order to further the work and test more of the model.
Many of these areas deal with the implementation of the model, and they will help to round out the capabilities of the tool, and ensure it covers a sufficiently sized design space.
Some of the areas are related more to tweaks to the implementation, not directly related to the model.
These tweaks will help to improve the overall function, and remove some of the buggier behavior of the implementation.
The areas that could result in the most striking improvements in creativity support deal with the interface, and are needed mostly to help users achieve a flow state.

\bodysubsection{Default Values}
\label{defaultValues}
In the current implementation, there are no default values assigned to any of the properties.
This results in there being no feedback to the user when they map properties that require multiple other properties to also be mapped.
Once the necessary subset of properties has been mapped, the primitive can draw itself, but in the intermediate time, the primitive does nothing.
For example, a rectangle primitive needs two horizontal properties mapped out of four, and two vertical properties mapped out of four.
Without default values, the primitive cannot draw itself until four out of those eight properties have been mapped.

To complicate things further, knowing which properties to assign defaults to depends on the internal geometric relationships of the primitive.
The best way to solve this is probably a decision tree that chooses which defaults to assign based on what else has been assigned already.
The important part of this decision tree structure is how it fits into the internal geometric relationship architecture, and how it gets initiated in the mapping assignment.

\bodysubsection{Primitives Graph}
\label{primitivesGraph}
The visualization primitives model allows for dependency relationships to occur between the properties of primitives.
These dependencies are represented through mappings.

Mappings in the current implementation record connections between data output, and a property.
The system for storing them is simplistic, as each only belongs to the parent primitive.
There are currently no methods for traversing the structures of mappings that can be created.
This presents problems with updating child primitives when a parent primitive is modified.

A proper system for storing mappings would allow traversal of the mappings, and prevent circular traversals when a loop is present in the structure.
The current implementation also means that scaling of values is clumsy when those values come from a parent primitive and not the original data.
A better structure would allow the properties to retrieve the scale value from the parent and modify it rather than just multiplying their own scale value on top.

\bodysubsection{Connections Between Primitives}
\label{primConnectionsFW}
Currently, the interface provides buttons for each visual property that a primitive has, and output buttons that pass the primitive's data through.
This interface works for testing the model, but it is hardly user-friendly.
Ideally, the interface should feel much more natural, and have the same kind of direct connections that come from the implementation's immediate feedback.

As mentioned, one possible way to accomplish that is through using the location of the prototypes relative to each other to help determine the spatial relationships between primitives.
This type of connection feels physical, and would be relatively intuitive to users.
However spatial relationships cannot define mappings for color or line weight.
These mappings sill need some sort of input and output interface, and may require things like icon representations or other interface elements to allow their mapping.
Inspiration for these elements could be drawn from layout or drawing tools.

\bodysubsection{Set of Primitives}
\label{setPrimitivesFW}
The set of primitives provided by the system has the most profound impact on what can be created with the system.
The current set is definitely not ideal.
Future tweaking of the set should help to optimize the balance between simplicity and flexibility.

One potential way to proceed with this is to build a list of visualizations that fall within the theoretical constraints of the system (tabular data, non-iterative layouts, etc.)
Establishing which visualizations in the list cannot be created with the current set of primitives will help to identify possible primitives that should be created.
It also may be possible to tweak the properties and internal geometric representations of existing primitives to allow the creation of other visualizations.

\bodysubsection{Creativity Support}
\label{creativitySupportFW}
The initial evaluation of creativity support had several issues.
Aside from using an earlier version of the CSI, the tutorial teaching people how to use the system probably needed to be enforced better.
Several people didn't finish the tutorial entirely before exploring on their own and continuing the study.

It would also be good to provide a task for participants to perform.
The participants of the last study seemed to just play around with the visuals, and not think much about 
A goal will help to focus people and give them constraints to work within.