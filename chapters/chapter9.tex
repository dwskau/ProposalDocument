\bodychapter{CONCLUSION}
\label{conclusion}

Our model presents a novel way to approach visualization design, and to support creativity.
Visualization primitives in conjunction with visual properties create an environment optimized for rapidly prototyping new visualization techniques.
The environment supports instant visual feedback and helps to develop an efficient flow for visualization design.

The theoretical contributions of this model are based on a cognitive argument that is closely aligned with visualization theory.
Our study shows considerable interest in and potential for the creation of new visualizations using visualization primitives.

The most compelling argument for this approach is the control and flexibility given to non-technical users who want to create novel visualizations.
We hope that some of the principles it presents can be integrated into existing and future visualization and visual analytics tools.

The planned future work will help to solidify the visualization primitives model as a useful way of conceptualizing data visualization.
The lessons learned from testing the limitations of the model will help to contextualize existing visualizations.
This can help to add to our theoretical understanding of visualization, especially with regards to how a visualization's visual characteristics impact what it is capable of displaying.
Improving the creativity support of the implementation will help to prove the  the model's capacity as a tool for successfully building data visualizations.
It will also help to understand the role of creativity in the visualization creation process, and perhaps help to determine what characteristics other visualization tools need to have to help support creativity.
