\bodychapter{RELATED WORK}
\label{relatedWork}
%Visualization primitives build on existing work on the theory of visualization, in particular Bertin's ideas and glyphs.
%We also consider them a tool, in particular one that is closely built on theory, like Protovis or D3.
%Finally, our goal is to encourage creative work with visualization, which requires ways of measuring creativity.

\bodysection{Theory}
\label{theory}

Information visualization is mostly understood as the visualization of discrete data points~\cite{tory2004rethinking}.
By contrast, Scientific visualization typically uses discretely sampled data that represents continuous data.
This distinction means that the visual representations can be treated as individuals for each data item.
In InfoVis, each data point is represented by a visual \textit{mark}, according to Bertin's highly influential work from the 1960s~\cite{bertin1983semiology}.
Bertin defined marks as graphical objects with visual (or \textit{retinal}) properties such as color, size, or position.

Mackinlay's APT system~\cite{Mackinlay1986} was directly based on Bertin's work, and the first one to create tailored visualizations for a dataset.
APT's goal was the automatic generation of the visualization, however, not specifically supporting the user's creativity.
SAGE~\cite{Roth:CHI:1994} provided similar functionality, but also primarily oriented at data rather than visual design.
Tableau~\cite{stolte2002polaris}, which is partly based on the ideas behind APT, goes in a similar direction, though the user can construct visualizations from scratch.
Tableau's expressive power is  limited to a relatively small number of plot types, however.
We want the user to be able to create entirely new types of visualizations quickly and easily.
In order to do this, a system that effectively supports creativity in the creation process may be necessary.

Card and Mackinlay developed a notation for analyzing the visual mappings used in visualizations~\cite{Card1997}.
Their goal was to connect point designs into a coherent design space, which then could be explored more effectively.
However, their system is mostly analytical, not constructive: changes in the analysis do not translate back into novel visualization designs.

The Grammar of Graphics~\cite{Wickham:JCGS:2010,Wilkinson2005c} describes a way of defining almost any visualization imaginable, but from a mathematical and computational perspective that does not necessarily match up with practical visualization creation or best practice.
For example, the idea that a pie chart is nothing but a stacked bar chart that has been transformed into polar coordinates may be a correct mathematical explanation, but we argue that few non-technical users would even understand this; much less construct a pie chart this way.

Visualization primitives are an extension of glyphs~\cite{anderson1957semigraphical}, which are graphical objects whose properties (lengths of parts, angles, colors) represent data.
Glyphs are typically used to show small numbers of high-dimensional data points, but we argue that the idea can be applied in a much more general way.
Tools for constructing glyphs~\cite{ribarsky1994glyphmaker} have been proposed, but tend to be driven by form-based interfaces and limited in the scope of available properties and ways of combining them into more complex (and ``traditional'') visualizations.

\bodysection{Tools}
\label{tools}

Visualization toolkits like prefuse~\cite{heer2005prefuse}, Protovis~\cite{bostock2009Protovis}, and D3.js~\cite{bostock2011d3} put many of Bertin's and Wilkinson's ideas into practice, and make it possible to create a wide variety of visualizations, including entirely new ones.
While they elegantly abstract away many common tasks, they do require considerable programming knowledge, however.
This is a major hurdle for many potential users of such tools, like journalists and designers.

Other tools that do not require programing, like Quadrigram~\cite{url:quadrigram,Ortiz2010} and DEFOG~\cite{Lins:TR:2011}, are still strongly influenced by computing constructs.
Building a visualization using Quadrigram's data flow paradigm requires many steps that are effectively function calls, and are based on typical operations available in programming languages.
This approach is conceptually different because it focuses the user on algorithms to reshape the data.
These algorithms are modifiable and adaptable, but still result in conceptualizing the visualization as a pipeline from original data to what gets drawn on screen.
DEFOG is also driven by data rather than the visual representation; that is a useful property for a data analysis tool, but it does not help designers construct new and interesting visualizations.

Several systems infer visualizations from hand drawn sketches, or abstractly created objects.
NapkinVis~\cite{Chao2010} infers a common visualization type and uses Protovis to draw the resulting charts.
CavePainting Visualizations by Keefe et al.~\cite{Keefe-2008-SSF} shows the value of having non-technical users involved in the visualization creation process, and explores how creativity can improve resulting visualizations.
Brett Victor has been developing a system for Drawing Dynamic Visualizations~\cite{url:drawingDynamic} that appears to infer data connections from the structure of what has been drawn.
These systems rely on expertise of technical users to create the actual visualizations by embedding it into the system, and by relying on known techniques.

On the other side of the spectrum, there are tools that allow users without programming experience to create visualizations.
Many Eyes~\cite{viegas2007manyeyes}, Excel, Adobe Illustrator, Tableau, all allow users to create data visualizations through simple input of data.
Unfortunately, the visualizations that these tools produce have all been predefined by the tool creator.
They each have a set number of visualizations types that they can create.
This constrains users from creating new visualizations, and possibly embedding domain knowledge or exposing new relationships within the visualization.

\bodysection{Creativity Support}
\label{creativitySupport}

There has not been a wide array of work exploring creativity in the creation of information visualizations, however some work does still involve creativity.
In the visual analytics space, it is widely acknowledged that creativity plays a role in the visual analytics workflow.~\cite{scholtz2006beyond}%todo: find more references to creativity in visual analytics
Lee et al.~\cite{lee2012beyond} have done some work investigating interaction with visualizations beyond keyboard and mouse.
Their design considerations very closely parallel design guidelines for supporting creativity.
Other significant work by Walny et al.~\cite{walny2011visual} explores visualization creation on whiteboards.
The study never explicitly investigates creativity, but the whiteboard format is relatively open ended, free-form, and collaborative, and the visualizations generated during the study are clearly creative.
CavePainting~\cite{Keefe-2008-SSF} intentionally incorporates creative users as a means to explore creative techniques applied to scientific visualization creation.
None of this work directly evaluates or explores creativity.

As creativity is a difficult concept to define, quantitatively evaluating creativity is also a difficult task.
The best option for evaluating creativity may be the Consensual Assessment Technique (CAT)~\cite{Amabile1996}.
The technique requires a panel of experts within the domain to evaluate the work, and rate its creativity.
The ratings revolve around how novel and useful the produced work is.

Evaluating a tool's support of creativity is somewhat easier, as it does not depend on the abilities or mental state of the user, only on the capabilities of the tool.
The Creativity Support Index (CSI)~\cite{carroll2009creativity} provides a system for measuring a tool's ability to support creativity.
It uses a set of six factors, ranked with Likert scales, followed by a set of fifteen ranked pairs to prioritize the factors.
The results of the index are a value out of 100, with 100 supporting creativity perfectly.
(For more on the CSI, refer to section~\ref{materials}.)

