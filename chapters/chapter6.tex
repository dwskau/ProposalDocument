\bodychapter{STUDY DESIGN}
\label{studyDesign}

The aim of the visualization primitives model is to enable creativity during the creation of data visualizations.
Given this goal, tests to measure the implementation need to evaluate the ability of the tool to support creativity.
This means that a user study will not evaluate visualization primitives against other tools using tests of task efficiency, or other similar metrics.
Instead, we employ the Creativity Support Index (CSI)~\cite{carroll2009creativity} to evaluate the tool's ability to support creativity.
The CSI is based closely on a tool�s ability to support a �flow state� for the user.

The study is intended to measure the tool�s ability to support creativity, not to quantify the quality of visualizations that are produced.
These concepts are related, however they are not identical.
Creative results are not guaranteed even with tools that support creativity perfectly.
Creativity depends on several components, including the mind of the creator, and the capabilities of the tool at hand.
We have included examples of ``creative'' visualizations generated with visualization primitives, however, these are anecdotal.
Evaluating how creative they are is a secondary issue to how well visualization primitives support creativity.

The nature of creativity requires that the user study be flexible about regulations.
Tasks that ask a user to find a certain thing in the data could provide inspiration, but they could also constrain thought processes.
Time limits impose pressure that could help or hinder the creative thought process.
We have opted to remove external pressures in the study, allowing participants to build visualizations at their own pace, setting their own tasks along the way.