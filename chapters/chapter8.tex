\bodychapter{FUTURE WORK}
\label{futureWork}

Putting the lessons learned into practice is the next step of the process.
The publication opportunities that are most appropriate for this work set the timeline for improvements to the interface and conducting studies.
The next feasible deadline is early December 2013.
To prepare for this deadline, and future deadlines beyond it, the interface needs to be given a lot of attention.
I would like to achieve all of the interface improvements in Section~\ref{interfaceImprovements}.
This puts the implementation on a good footing to have several studies run against it with only minor changes between each study.

The study design needs to be adjusted heavily to account for the addition of the CAT, and to encourage participants to complete the tasks more thoroughly.
Providing a task will help with completion, but it may be even more important to run the study with individuals in a lab setting rather than over the internet.
This also makes it simpler to monetarily reward participants for their participation, and provide further motivation.
Using contacts in the visualization domain, I anticipate access to a handful of students studying visualization, as well as a small collection of information designers working in the visualization space.
This group provides a set of participants that are likely familiar with data visualization concepts, with a high likelihood of being able to perform creative tasks using the tool.

Studies beyond this one will be based somewhat on the results of the study, but the main goals will be to establish the creativity support of the visualization primitives model.
Another possible angle to examine is whether the model supports creativity better because of how closely it matches the mental model of visualization creators.
If this turns out to be true, then the work will make a significant contribution to our understanding of information visualization, and can be used in the development of visualization theory.