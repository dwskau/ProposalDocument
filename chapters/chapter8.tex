\bodychapter{DISCUSSION}
\label{discussion}

The results of the experiment indicate that visualization primitives support creativity, especially with users who have had adequate experience with visualization tools.

User interface issues were a limiting factor in users' exploration of the design space.
Specifically, having to mouse over properties to see what they were bound to made it difficult to build and maintain a mental model of what was being shown.
While our goal was to reduce clutter from connecting lines, the alternative may have created more issues than it solved.

The unfamiliarity of the interface definitely contributed to many users frustrations.
Many comments expressed users' frustration with not knowing what the parts of the interface did.
This could be addressed by further iterations of the tutorial.

The internal geometric representations of the primitives also may have contributed to confusion.
Drawing elements to the screen is often not possible until multiple properties have been specified.
This could be alleviated with default values, however the solution is far from elegant, and can produce unexpected behavior when assigning properties.

Not all the results that participants created and submitted were strictly visualizations in the sense of being useful. We see some playful examples that are more artistic than useful, which is in line with our goals. We did not ask people to create anything in particular so as not to constrain their exploration.

The considerable time spent on the site shows that they were engaged and interested in exploration, despite the user interface flaws This suggests that tapping into users' creativity is a promising way of getting them interested in visualization.