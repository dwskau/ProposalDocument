\bodychapter{COGNITIVE MOTIVATION}
\label{cognitiveMotivation}

Part of the motivation for this work, both for the theoretical side and the practical uses for visualization designers, is based on the idea of designing visual objects as actual objects (rather than a pipeline).
We believe that this is approach more closely matches an effective mental model that can benefit both education in information visualization and creativity in creating visualizations.

\bodysection{The Information Visualization Model}
\label{infoVisModel}

The theory of information visualization is built on discrete data items.
Bertin's marks are individual objects, just as the data points rendered by APT~\cite{Mackinlay1986} and the data items fed to a piece of Protovis~\cite{bostock2009Protovis} code.
One of the two dimensions Tory and M\"oller~\cite{tory2004rethinking} use to distinguish between scientific and information visualization is also the discrete domain of the data (the other dimension being whether the spatial layout is given or chosen).

This distinction is fundamental and important, because it leads to entirely different types of solutions than work being done in scientific visualization.
Ray casting, splatting, and other techniques that assume and require a continuous data domain (where values can be meaningfully interpolated between data points) are of no use in information visualization; though some InfoVis techniques are applicable in scientific visualization, because any real dataset necessarily consists of a limited number of data values.

Distinct data values translate into distinct, well-defined visual objects.
While a volume dataset can be rendered at many different resolutions, and can even be improved by better reconstruction and interpolation techniques, the number of items to be drawn for a given dataset in information visualization is not up for debate: one visual shape for each data point (except when there is filtering or aggregation).

This concept of maintaining the same discretization throughout the path from data to visual representation is important.
It helps the simplicity of the model, and makes it possible to connect individuals in the visualization to individuals in the data source.
It also means that an object based approach to creating data visualizations is a good fit.
In particular, prefuse~\cite{heer2005prefuse} and Protovis~\cite{bostock2009Protovis} provide classes that closely resemble primitives, as well as the means to iterate over such definitions to render data onto the screen.

\bodysection{Cognitive Models}
\label{cognitiveModel}

Our perception is based on the notion of objects.
Very early on, infants develop the ability to differentiate and track objects~\cite{Spelke:CogSci:1990}.
We do not see our world as a collection of pixels or surfaces, but distinct, physical objects.

This also applies to our perception of two-dimensional shapes.
Pinker~\cite{Pinker:AIFT:1990} describes a model of the mental processes behind reading and understanding a chart as an exercise in separating it into objects and examining the relationships between them.
The user translates the properties of all these objects on the chart into cognitive representations that answer conceptual questions.
Pinker shows that the cognitive representations of these objects are symbols with visual descriptions.

Recent work on the perceived relationships between elements of visualizations~\cite{Ziemkiewicz:InfoVis:2010} has reinforced this idea: items in a bubble chart seemed to attract each other when they were similar or close to one another.
Such an effect is only possible if these elements are, in fact, seen as actual objects, with mass and other physical properties.

Grammel et al.~\cite{Grammel2010} calls for tools supporting iterative refinement, as well as tools to help teach novices about visualization.
The model we propose can support tools that have iterative refinement.
We also believe the experimentation process that this model allows can help users to learn about information visualization.

We believe that having a model that closely aligns with our perceptual and cognitive processes can assist creativity by allowing users to think in more natural ways.
As a result, it may help to support creativity when using tools based on the model.