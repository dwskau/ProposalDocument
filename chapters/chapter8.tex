\bodychapter{FUTURE WORK}
\label{futureWork}

Putting the lessons learned into practice is the next step of the process.
The publication opportunities that are most appropriate for this work set the timeline for improvements to the interface and conducting studies.
The interface needs to be given a lot of attention.
I would like to achieve all of the interface improvements in Section~\ref{interfaceImprovements}.
This puts the implementation on a good footing to have several studies run against it with only minor changes between each study.

The study design needs to be adjusted heavily to account for the addition of the CAT, and to encourage participants to complete the tasks more thoroughly.
Providing a task will help with completion.
One possible task would be showing three visualizations that show airlines that have fewer delays, using an Airline On-Time Performance dataset.
Another possible task is creating three visualizations that show data dimensions that correlate with miles per gallon in a dataset on cars.
A concrete task will provide participants with boundaries and restrictions, giving them restrictions to be creative within.
The task will also make it possible to judge how creative their solutions are.

It may also be important to run the study with individuals in a lab setting rather than over the internet.
This also makes it simpler to monetarily reward participants for their participation, and provide further motivation (funding provided by Visually).
I can ensure that the participant fully understands the interface, and has an opportunity to ask questions before they are left to the task.
It also means browser compatibility will not be an issue.

Using contacts in the visualization domain, I have access to a handful of students studying visualization, as well as a large collection of information designers working in the visualization space.
The student members of this group may be pulled from data visualization classes at UC Berkeley, and the designers may be pulled from a pool of contacts I have through my work at Visually.
This group provides a set of participants that are likely familiar with data visualization concepts, and able to perform creative tasks using the tool.

The study will test two different interfaces.
One interface will be visualization primitives, while the other will be Tableau Public.
Tableau is the closest available software to the visualization primitives model and allows the most freeform creation of visualizations.
Both interfaces will have the same data set loaded, and the tasks will be the same in both cases.
Participants will experience both interfaces, in random order.

The visualization primitives portion will begin with an information session on the tool and the components of its interface.
Participants will be trained on how to make a visualization, and how to connect primitives together.
After training, they will be presented with the tasks, and asked to solve them.
Following that, they will be given a time period of free exploration where they can explore the interface at their will, and ask questions freely.
During this stage, they will be allowed to designate any of the things they produce as "creative".
Once they have completed the assignment, they will be given the CSI to answer for the visualization primitives interface.

The Tableau portion will follow mostly the same model.
It will begin with an information session on how to use Tableau Public, and the components of the interface.
Participants will be trained on how to make and manipulate a visualization, before being presented with the tasks and asked to solve them.
This will also be followed by a period of free exploration with an opportunity to mark produced things as creative.
After completion, they will be given the CSI again to answer for the Tableau interface.
Once participants have used both interfaces, their participation in the study is over.

The expert judges will also be drawn from my contacts through Visually.
They will be chosen based on their expertise in visualization, and will come from both the academic realm (professors of data visualization), and from practicing leaders in the field.
They will be monetarily rewarded for their participation in the study (funding provided by Visually).
An interface will be provided to allow the judges to rate the graphics over the internet.
The judges will be trained, either in person, or over a video call, on the rating system, and on how to use the interface rate each participant produced visualization.
The judges will be instructed to do these activities alone, without consultation with others, to ensure they do not influence each other.

I expect to find that the visualization primitives interface allows for more creative results than Tableau.
I expect it to have a higher CSI score, and I expect the judges to rate the products as more creative.
This would mean that the approach provides better freedom, and better support of creativity in the creation of data visualizations.

Studies beyond this one will be based somewhat on the results of the study, but the main goals will be to establish the creativity support of the visualization primitives model.
Another possible angle to examine is whether the model supports creativity better because of how closely it matches the mental model of visualization creators.
If this turns out to be true, then the work will make a significant contribution to our understanding of information visualization, and can be used in the development of visualization theory.